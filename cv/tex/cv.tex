\documentclass{article}

\usepackage{hyperref}
\usepackage{geometry}
\usepackage{graphicx}

\hypersetup{
  colorlinks = true,
  urlcolor = blue,
  pdfauthor = {Christoph Junghans},
  pdfkeywords = {computational physics, monte carlo, molecular dynamics,votca},
  pdftitle = {Christoph Junghans: Curriculum Vitae},
  pdfsubject = {Curriculum Vitae},
  pdfpagemode = UseNone
}

\geometry{
  body={6.5in, 8.5in},
  left=1.0in,
  top=1.25in
}

% Customize page headers
\pagestyle{myheadings}
\markright{Christoph Junghans}
\pagestyle{empty}

% Custom section fonts
\usepackage{sectsty}
\sectionfont{\rmfamily\mdseries\large\bf}
\subsectionfont{\rmfamily\mdseries\itshape\normalsize}

% Don't indent paragraphs.
\setlength\parindent{0em}

\begin{document}

% Place name at left
\begin{tabular}{p{\textwidth}}
{\huge Christoph Junghans}\\
\hline
\end{tabular}
\vspace{0.25in}

\begin{minipage}{0.45\linewidth}
  Postdoctoral research assistant\\
  Theory of Polymers\\
  \href{http://www.mpip-mainz.mpg.de}{Max Planck Institute for Polymer Research} \\
  Mainz, Germany
\end{minipage}
\begin{minipage}{0.45\linewidth}
  \begin{tabular}{ll}
    Phone: & ++49-(0)6131-379-335 \\
    Email: & \href{mailto:junghans@mpip-mainz.mpg.de}{\tt junghans@mpip-mainz.mpg.de} \\
    Homepage: & \href{http://www.comppyhs.de}{\tt http://www.compphys.de} \\
  \end{tabular}
\end{minipage}

\section*{Research Interests}
\begin{itemize}
\setlength{\itemsep}{0pt}
\setlength{\parskip}{0pt}
\setlength{\parsep}{0pt}
\item Development of adaptive multiscale simulation approaches
\item Coarse-graining of liquids and implementation of new methodologies
\item Software design and management -- coarse-graining package \htmladdnormallink{VOTCA}{http://www.votca.org}
\item Generalized ensemble methods to study phase transitions
\item Porting and optimization of scientific codes (\htmladdnormallink{VOTCA}{http://www.votca.org}, \htmladdnormallink{ESPReSo}{http://www.espressomd.org}, \htmladdnormallink{GroMaCS}{http://www.gromacs.org})
\item \textbf{Currently:} implementation of long-range electrostatic methods into the adaptive resolution scheme
\end{itemize}

\section*{Education}

\begin{tabular}{ll}
Oct 2010& \textbf{Ph.D. in Physics}, Johannes Gutenberg University of Mainz, Germany\\
&\htmladdnormallink{\textit{Between the Scales: Water from different Perspectives}}{http://www.mpip-mainz.mpg.de/~junghans/publications/2010e.pdf} -- An adaptive resolution Molecular Dynamics\\
&study using several Coarse-Graining methods implement\\
Nov 2006&\textbf{Diploma in Physics}, University of Leipzig, Germany\\
&\htmladdnormallink{\textit{Aggregation of Mesoscopic Protein-like Heteropolymers}}{http://www.mpip-mainz.mpg.de/~junghans/publications/2006d.pdf}-- A generalized ensemble Monte Carlo study\\
\end{tabular}

\section*{Work Experience}
\begin{tabular}{lcll}
Nov 2010 &-& Present & \textbf{Postdoctoral research assistant}, Max Planck Institute for Polymer Research\\
Jan 2007 &-& Oct 2010 & \textbf{PhD Student}, Max Planck Institute for Polymer Research, Mainz\\
Jan 2009 &-& July 2009 & \textbf{Specialist for Application Performance} \& Deep Computing,\\
&&&IBM Systems \& Technology Group Europe \\
Oct 2003 &-& Sept 2006 & \textbf{Student assistant}, Institute for Theoretical Physics, University of Leipzig \\
Aug 2005 &-& Oct 2005 & \textbf{Research student} in the ``Scientific Computing'' program,\\
&&& J{\"u}lich Supercomputing Centre \\
\multicolumn{3}{l}{Sept 2004} & \textbf{Student assistant}, Institute for Meteorology, University of Leipzig \\
Feb 2003 &-& April 2003 & \textbf{Research student}, Chair for Fluid Mechanics,\\
&&&Martin Luther University of Halle-Wittenberg \\
\end{tabular}

\section*{Awards}

\begin{tabular}{ll}
June 2010 & Participant of 60th Lindau Nobel Laureate Meetings\\
Nov 2005  & Wolfgang Natonek award, University of Leipzig\\
Oct 2003  & Teubner award, Department for Physics and Earth Science, University of Leipzig\\
\end{tabular}

\section*{Open Source Projects}

\begin{tabular}{p{0.18\textwidth}p{0.55\textwidth}p{0.2\textwidth}}
VOTCA & A toolkit for coarse-graining applications & Core Developer \\
ESPReSo & Extensible Simulation Package for Research on Soft matter & Developer \\
GroMaCS & A versatile package to perform molecular dynamics & Developer \\
ESPReSo++ & Successor of the ESPResSo simulation package & Developer \\
\end{tabular}\\
For more information can be found on \htmladdnormallink{my ohloh profile}{http://www.ohloh.net/accounts/junghans}.

\section*{Computer Skills}

\begin{tabular}{ll}
\textbf{General} & Linux, AIX, Mac OS, DOS, Windows \\
\textbf{Programming} & C, Fortran, MPI, OpenMP, C++ \\
\textbf{Scripting} & Shell, awk, Perl, expect, PHP, tcl \\
\textbf{Markup Languages} & HTML, latex, mediawiki, txt2tags \\
\end{tabular}

\section*{Teaching Experience}

\begin{tabular}{lcll}
April 2007 &-& Sept 2008 & Tutor, Theoretical physics I/II/III, Johannes Gutenberg University of Mainz \\
April 2006 &-& Sept 2006 & Tutor, Introduction to computer simulations I, University of Leipzig \\
Oct 2005 &-& March 2006 & Tutor, Simulation methods in generalized ensembles, University of Leipzig \\
Sept 2004 &-& Oct 2004 & Tutor, Mathematical preparation course for first-year students, University of Leipzig \\
\end{tabular}

\section*{Representative Experience}

\begin{tabular}{p{0.3\textwidth}p{0.58\textwidth}}
July 2007 - Dec 2008 & PhD representative of the Kremer group, Max Planck Institute for Polymer Research, Mainz \\
Nov 2007 - Nov 2008 & PhD representative of the MPI for Polymer Research \\
Oct 2002 - Sept 2005 & Elected member of student government (Fachschaftsrates) of the Department for Physics and Earth Science, University of Leipzig - Extensive committee work in this period\\
\end{tabular}

\section*{Invited Talks}

\begin{tabular}{p{0.2\textwidth}p{0.73\textwidth}}
Oct 14th 2010 & ``Multi-scale modeling using AdResS'', CECAM Workshop ``Simulating Soft Matter with ESPResSo'', ICP Stuttgart\\
May 31th 2010 & Votca Workshop, CSI Darmstadt \\
Nov 19th 2009 & ``Versatile object-oriented toolkit for coarse-graining applications'', Computational Biology Cluster Seminar, IFF-2, FZ J{\"u}lich\\
Nov 27th 2008 & ``Comparative atomistic and coarse-grained study of water: simulation details vs. simulation feasibility'', CompPhys08, ITP Leipzig\\
Nov 30th 2007 & ``Controlling material properties using a thermostat'', CompPhys07, ITP Leipzig\\
\end{tabular}

\section*{Publications}

\subsection*{Reviewed Papers}

\begin{enumerate}
\item[11.] C. Junghans, M. Bachmann and W. Janke,
  \textit{Hierarchies in Nucleation Transitions},
  \htmladdnormallink{Comp. Phys. Comm., in press (2010)}{http://dx.doi.org/10.1016/j.cpc.2010.11.015}.

\item[10.] B. P. Lambeth, Jr., C. Junghans, K. Kremer, C. Clementi, and L. Delle Site, 
  \textit{Communication: On the Locality of Hydrogen Bond Networks at Hydrophobic Interface},
  \htmladdnormallink{J. Chem. Phys. \textbf{133},  221101 (2010)}{http://dx.doi.org/10.1063/1.3522773}.

\item[9.] C. Junghans and S. Poblete,
  \textit{A reference implementation of the adaptive resolution scheme in ESPResSo},
  \htmladdnormallink{Comp. Phys. Comm. \textbf{181}, 1449 (2010)}{http://dx.doi.org/10.1016/j.cpc.2010.04.013}.

\item[8.] V. R{\"u}hle, C. Junghans, A. Lukyanov, K. Kremer and D. Andrienko,
  \textit{Versatile Object-oriented Toolkit for Coarse-graining Applications},
  \htmladdnormallink{J. Chem. Theo. Comp. \textbf{5}, 3211 (2009)}{http://dx.doi.org/10.1021/ct900369w}. 

\item[7.] C. Junghans, M. Bachmann and W. Janke,
  \textit{Statistical Mechanics of Aggregation and Crystallization for Semiflexible Polymers},
  \htmladdnormallink{Europhys. Lett. \textbf{87}, 40002 (2009)}{http://dx.doi.org/10.1209/0295-5075/87/40002}.

\item[6.] H. Wang, C. Junghans and K. Kremer,
  \textit{Comparative atomistic and coarse-grained study of water: What do we lose by coarse-graining?},
  \htmladdnormallink{Euro. Phys. J. E \textbf{28}, 221 (2009)}{http://dx.doi.org/10.1140/epje/i2008-10413-5}.

\item[5.] M. Praprotnik, C. Junghans, L. Delle Site and K. Kremer,
  \textit{Simulation approaches to soft matter: Generic statistical properties vs. chemical details},
  \htmladdnormallink{Comp. Phys. Comm. \textbf{179}, 51 (2008)}{http://dx.doi.org/10.1016/j.cpc.2008.01.018}.

\item[4.] C. Junghans, M. Bachmann and W. Janke,
  \textit{Thermodynamics of Peptide Aggregation Processes: An Analysis from Perspectives of Three Statistical Ensembles},
  \htmladdnormallink{J. Chem. Phys. \textbf{128}, 085103 (2008)}{http://dx.doi.org/10.1063/1.2830233} .
  Also featured in: \htmladdnormallink{Virt. J. Nanoscale Sci. \& Techn. \textbf{17}(10) (2008)}{http://scitation.aip.org/dbt/dbt.jsp?KEY=VIRT01\&Volume=17\&Issue=10} and \htmladdnormallink{Virt. J. Biol. Phys. Res. \textbf{15}(5) (2008)}{http://ojps.aip.org/dbt/dbt.jsp?KEY=VIRT02\&Volume=15\&Issue=5}.

\item[3.] C. Junghans, M. Praprotnik and K. Kremer,
  \textit{Transport properties controlled by a thermostat: An extended dissipative particle dynamics thermostat},
  \htmladdnormallink{Soft Matter \textbf{4}, 156 (2008)}{http://dx.doi.org/10.1039/b713568h}.

\item[2.] C. Junghans, M. Bachmann and W. Janke,
  \textit{Microcanonical Analyses of Peptide Aggregation Processes},
  \htmladdnormallink{Phys. Rev. Lett. \textbf{97}, 218103 (2006)}{http://dx.doi.org/10.1103/PhysRevLett.97.218103}.

\item[1.] C. Junghans and U. H. E. Hansmann,
  \textit{Numerical Comparison of Wang Landau Sampling and Parallel Tempering for Met-enkephalin}, 
  \htmladdnormallink{Int. J. Mod. Phys. C \textbf{17}, 817 (2006)}{http://dx.doi.org/10.1142/S012918310600931X}.
\end{enumerate}

More infomation can be found on \htmladdnormallink{my ResearcherID profile}{http://www.researcherid.com/rid/G-4238-2010}.

\subsection*{Book Chapters}

\begin{enumerate}
\item[1.] C. Junghans, M. Praprotnik and L. Delle Site,
  \textit{Adaptive Resolution Schemes},
  in: J. Grotendorst, N. Attig, S. Bl{\"u}gel and D. Marx (Eds.),
  Multiscale Simulation Methods in Molecular Sciences, \htmladdnormallink{NIC Series Vol. 42}{http://www.fz-juelich.de/nic-series/volume42/nic-series-volume42.pdf}, J{\"u}lich (2009), 359.
\end{enumerate}

\subsection*{References}

\begin{minipage}{0.45\linewidth}
Prof.\ K.\ Kremer
\vspace{-2mm}\\
{\tiny PhD Advisor}\\
\href{http://www.mpip-mainz.mpg.de}{Max Planck Institute for Polymer Research} \\
Email: \href{mailto:kremer@mpip-mainz.mpg.de}{\tt kremer@mpip-mainz.mpg.de}
\end{minipage}
\begin{minipage}{0.45\linewidth}
Prof.\ W.\ Janke
\vspace{-2mm}\\
{\tiny Diploma Advisor}\\
\href{http://www.physik.uni-leipzig.de/~janke/}{ITP, University of Leipzig} \\
Email: \href{mailto:janke@itp.uni-leipzig.de}{\tt janke@itp.uni-leipzig.de}
\end{minipage}

\vspace{2ex}
\begin{minipage}{0.45\linewidth}
Dr.\ B.\ Hess
\vspace{-2mm}\\
{\tiny Gromacs Developer}\\
\href{http://www.sbc.su.se/}{SBC, Stockholm University} \\
Email: \href{mailto:hess@cbr.su.se}{\tt hess@cbr.su.se}
\end{minipage}
\begin{minipage}{0.45\linewidth}
Dr.\ M. P{\"u}tz
\vspace{-2mm}\\
{\tiny Application Performance Specialist}\\
IBM Germany \\
Email: \href{mailto:mpuetz@de.ibm.com}{\tt mpuetz@de.ibm.com}
\end{minipage}
\end{document}


